% Options for packages loaded elsewhere
\PassOptionsToPackage{unicode}{hyperref}
\PassOptionsToPackage{hyphens}{url}
%
\documentclass[
]{article}
\usepackage{amsmath,amssymb}
\usepackage{lmodern}
\usepackage{iftex}
\ifPDFTeX
  \usepackage[T1]{fontenc}
  \usepackage[utf8]{inputenc}
  \usepackage{textcomp} % provide euro and other symbols
\else % if luatex or xetex
  \usepackage{unicode-math}
  \defaultfontfeatures{Scale=MatchLowercase}
  \defaultfontfeatures[\rmfamily]{Ligatures=TeX,Scale=1}
\fi
% Use upquote if available, for straight quotes in verbatim environments
\IfFileExists{upquote.sty}{\usepackage{upquote}}{}
\IfFileExists{microtype.sty}{% use microtype if available
  \usepackage[]{microtype}
  \UseMicrotypeSet[protrusion]{basicmath} % disable protrusion for tt fonts
}{}
\makeatletter
\@ifundefined{KOMAClassName}{% if non-KOMA class
  \IfFileExists{parskip.sty}{%
    \usepackage{parskip}
  }{% else
    \setlength{\parindent}{0pt}
    \setlength{\parskip}{6pt plus 2pt minus 1pt}}
}{% if KOMA class
  \KOMAoptions{parskip=half}}
\makeatother
\usepackage{xcolor}
\usepackage[margin=1in]{geometry}
\usepackage{graphicx}
\makeatletter
\def\maxwidth{\ifdim\Gin@nat@width>\linewidth\linewidth\else\Gin@nat@width\fi}
\def\maxheight{\ifdim\Gin@nat@height>\textheight\textheight\else\Gin@nat@height\fi}
\makeatother
% Scale images if necessary, so that they will not overflow the page
% margins by default, and it is still possible to overwrite the defaults
% using explicit options in \includegraphics[width, height, ...]{}
\setkeys{Gin}{width=\maxwidth,height=\maxheight,keepaspectratio}
% Set default figure placement to htbp
\makeatletter
\def\fps@figure{htbp}
\makeatother
\setlength{\emergencystretch}{3em} % prevent overfull lines
\providecommand{\tightlist}{%
  \setlength{\itemsep}{0pt}\setlength{\parskip}{0pt}}
\setcounter{secnumdepth}{-\maxdimen} % remove section numbering
\ifLuaTeX
  \usepackage{selnolig}  % disable illegal ligatures
\fi
\IfFileExists{bookmark.sty}{\usepackage{bookmark}}{\usepackage{hyperref}}
\IfFileExists{xurl.sty}{\usepackage{xurl}}{} % add URL line breaks if available
\urlstyle{same} % disable monospaced font for URLs
\hypersetup{
  pdftitle={Estatística I},
  hidelinks,
  pdfcreator={LaTeX via pandoc}}

\title{Estatística I}
\usepackage{etoolbox}
\makeatletter
\providecommand{\subtitle}[1]{% add subtitle to \maketitle
  \apptocmd{\@title}{\par {\large #1 \par}}{}{}
}
\makeatother
\subtitle{Lista 05 - Distribuições de Probabilidades Discretas\\
Gabarito}
\author{}
\date{\vspace{-2.5em}}

\begin{document}
\maketitle

\hypertarget{distribuiuxe7uxe3o-de-probabilidade-discreta}{%
\section{Distribuição de Probabilidade
Discreta}\label{distribuiuxe7uxe3o-de-probabilidade-discreta}}

\hypertarget{questuxe3o-1}{%
\subsection{Questão 1}\label{questuxe3o-1}}

VERDADEIRO, FALSO, VERDADEIRO, FALSO, FALSO, VERDADEIRO, FALSO,
VERDADEIRO, VERDADEIRO, FALSO

\hypertarget{questuxe3o-2}{%
\subsection{Questão 2}\label{questuxe3o-2}}

\begin{enumerate}
\def\labelenumi{\alph{enumi})}
\item
  \(b=2\)
\item
  \(b=2k\)
\end{enumerate}

\hypertarget{questuxe3o-3}{%
\subsection{Questão 3}\label{questuxe3o-3}}

VERDADEIRO, FALSO

\hypertarget{questuxe3o-4}{%
\subsection{Questão 4}\label{questuxe3o-4}}

Binomial. \(P(X=8)=0,1201\)

\hypertarget{questuxe3o-5}{%
\subsection{Questão 5}\label{questuxe3o-5}}

\begin{enumerate}
\def\labelenumi{\alph{enumi})}
\item
  \(P(X=5)=\) 0.2023312
\item
  \(P(X\geq 3) = 1-P(X<3)=\) 0.9087396
\item
  \(P(X=0)=\) 0.0031712
\end{enumerate}

\hypertarget{questuxe3o-6}{%
\subsection{Questão 6}\label{questuxe3o-6}}

Distribuição de Poisson

\begin{enumerate}
\def\labelenumi{\alph{enumi})}
\item
  \(f_{3h}(k=10|gripada)=\frac{9^{10} e^{-9}}{10!}=\) 0.1185801
\item
  \(f_{2h}(k\geq 1|gripada)=1-f_{2h}(k=0|gripada)=\) 0.9975212
\item
  \(P_{1h}(gripada|3 espirros) = \frac{P_{1h}(3 espirros|gripada)P_{1h}(gripada)}{P_{1h}(3 espirros)}\).
  Teorema de Bayes
\end{enumerate}

Logo \(P_{1h}(gripada|3 espirros) =\) 0.7851335.

Reforçando a importância do Teorema de Bayes. Como vimos nesse problema,
sem saber que a aluna espirrou 3 vezes (ou seja, a \emph{priori}),
sabemos que ela tem 0,5 de probabilidade de estar gripada. No momento em
que recebemos a informação de que ela espirrou 3 vezes, podemos
\textbf{atualizar} nossa crença de acordo com a nova informação e
calcular a probabilidade a \emph{posteriori}. Para tanto, utilizamos o
Teorema de Bayes, que é o pilar da Econometria Bayesiana. Recomendo o
vídeo do canal \texttt{3blue1brown} sobre Teorema de Bayes que está
neste \href{https://www.youtube.com/watch?v=HZGCoVF3YvM\&t=471s}{link}.

\hypertarget{questuxe3o-7}{%
\subsection{Questão 7}\label{questuxe3o-7}}

\begin{enumerate}
\def\labelenumi{\alph{enumi})}
\item
  Para \(Y = y\), precisamos de \(y − 1\) fracassos antes de
  conseguirmos um acerto. Como um fracasso tem probabilidade \((1 − p)\)
  e os eventos são independentes, temos: \(P(Y = y) = p(1 − p)^{y−1}\)
\item
  \(M_Y(t)=\frac{pe^t}{1-e^t(1-p)}\)
\item
  \(M'(t)=\frac{pe^t}{(1-e^t+pe^t)^2}\) e \(E[Y]=M'(0)\). Logo:
\end{enumerate}

\(E[Y]=\frac{pe^0}{(1-e^0+pe^0)^2}=1/p\)

\begin{enumerate}
\def\labelenumi{\alph{enumi})}
\setcounter{enumi}{3}
\tightlist
\item
  \(\frac{d^2 M}{d t^2}=\frac{-pe^t[(p-1)e^t-1]}{[(p-1)e^t+1]^3}.\)
\end{enumerate}

\(Var[Y]=M''(0)-E[Y]^2=\frac{2-p}{p^2}-\frac{1}{p^2}=\frac{1-p}{p^2}\)

\end{document}
